% ----------------------------------------------------------------
% AMS-LaTeX Paper ************************************************
% **** -----------------------------------------------------------
%\documentclass[14pt, openany, twoside]{book}
\documentclass[12pt, openany, oneside]{book}
%\usepackage[14pt]{extsizes}
\textwidth=16cm \oddsidemargin=0pt \evensidemargin=0pt
\usepackage{amsmath,amsthm,amssymb}
\let\amslrcorner\lrcorner
\usepackage[T1,T2A]{fontenc}
%\usepackage[cp1251]{inputenc}
\usepackage[english, russian]{babel}
\usepackage{amscd}
\usepackage{oldgerm}
\usepackage[matrix,arrow,curve]{xy}
\usepackage{tikz-cd}
\usepackage{comment}
\usepackage{xcolor}
\usepackage{hyperref}
\usepackage{mathtools}
\usepackage{extpfeil} 
\usepackage{cmap}
\usepackage{enumitem}
\usepackage{amsmath}
\usepackage{tensor}
\usepackage{amssymb}
\usepackage{graphicx}
\usepackage{subfig}
\usepackage{derivative}
%\usepackage[left=3cm,right=1.5cm,top=2cm,bottom=2cm]{geometry}
\usepackage[left=30mm, top=20mm, right=30mm, bottom=20mm,nohead, includefoot,footskip=35pt]{geometry}
%\usepackage[left=30mm, top=20mm, right=15mm, bottom=20mm, nohead, includefoot,footskip=35pt]{geometry}
% Межстрочный интервал = 1.5pt
%\usepackage{setspace}
%\onehalfspacing

% Абзацный отступ = 1.25см
\usepackage{indentfirst}
\setlength\parindent{12.5mm}{\tiny }

\usepackage{fancyhdr}
\pagestyle{plain}
\fancyhf{}
\fancyfoot[R]{\centering\thepage}

\usepackage{amscd}
\usepackage{oldgerm}
\usepackage[matrix,arrow,curve]{xy}
\usepackage{tikz-cd}
\usepackage{comment}
%\usepackage{xcolor}
\usepackage{hyperref}
\usepackage{mathtools}
\usepackage{extpfeil} 
\usepackage{tikz}
\usetikzlibrary{patterns}
\usetikzlibrary{babel}
\definecolor{linkcolor}{HTML}{799B03} % цвет ссылок
\definecolor{urlcolor}{HTML}{799B03} % цвет гиперссылок
\hypersetup{pdfstartview=FitH,  linkcolor=linkcolor,urlcolor=urlcolor, colorlinks=true}
\vfuzz2pt % Don't report over-full v-boxes if over-edge is small
\hfuzz2pt % Don't report over-full h-boxes if over-edge is small
% THEOREMS -------------------------------------------------------
\newtheorem{thm}{Теорема}[section]
\newtheorem{exercise}[thm]{Упражнение}
\newtheorem*{thm1}{Теорема}
\newtheorem{cor}[thm]{Следствие}
\newtheorem{lem}[thm]{Лемма}
\newtheorem{zad}[thm]{Задача}
\newtheorem{prop}[thm]{Предложение}
\newtheorem{prop*}[thm]{*Предложение}
\theoremstyle{definition}
\newtheorem{dfn}[thm]{Определение}
\newtheorem{lemma}{Лемма} % чтобы заработало окружение лемм
%\newtheorem{dfn}{Определение}[section]
\newtheorem{dfn*}{*Определение}[section]
\newtheorem*{defn1}{Определение}
\theoremstyle{remark}
\newtheorem{rem}[thm]{Замечание}
\newtheorem{example}{Пример}[section]
\numberwithin{equation}{section}
\renewcommand{\proofname}{Доказательство}
\numberwithin{equation}{section}

% MATH -----------------------------------------------------------
\newcommand{\set}[1]{\left\{#1\right\}}
\newcommand{\gen}[1]{\left\langle#1\right\rangle}
\newcommand{\dpr}[2]{\left(#1\shortmid#2\right)}% Dot product
\newcommand{\eps}{\varepsilon}
\newcommand{\rank}{\mathrm{rank}}
\newcommand{\coker}{\mathrm{coker}}
\newcommand{\Hom}{\mathrm{Hom}}
\newcommand{\codim}{\mathrm{codim}}
\newcommand{\To}{\longrightarrow}
\newcommand{\0}{\varnothing}
\newcommand{\N}{\mathbb{N}}
\newcommand{\Z}{\mathbb{Z}}
\newcommand{\Q}{\mathbb{Q}}
\newcommand{\CN}{\mathbb{C}}
\newcommand{\HN}{\mathbb{H}}
\newcommand{\End}{\mathrm{End}}
\newcommand{\M}{\mathrm{M}}
\newcommand{\R}{\mathbb{R}}
\newcommand{\RN}{\mathbb{R}}
\newcommand{\tr}{\mathrm{t}}
\newcommand{\trace}{\mathrm{tr}}
\newcommand{\Aut}{\mathrm{Aut}}
\newcommand{\spec}{\mathrm{Spec}}
\newcommand{\diag}{\mathrm{diag}}
\newcommand{\GL}{\mathrm{GL}}
\newcommand{\SU}{\mathrm{SU}}
\newcommand{\OG}{\mathrm{O}}
\newcommand{\SO}{\mathrm{SO}}
\newcommand{\m}{\mathrm{M}}
\newcommand{\un}{\mathrm{U}}
\newcommand{\UG}{\mathrm{U}}
\newcommand{\re}{\operatorname{Re}}
%\DeclareMathOperator{\re}{re}
\newcommand{\im}{\operatorname{Im}}
\newcommand{\ovl}{\overline}
\newcommand{\norm}[1]{\left\lVert#1\right\rVert}
\newcommand{\mf}[1]{{\mathfrak{#1}}}
\newcommand{\id}{\operatorname{id}}
\newcommand{\sm}{\setminus\set{0}}
\newcommand{\PS}{\operatorname{P}}
\DeclareMathOperator{\noreq}{\trianglelefteq}
\newcommand{\Co}{\mathbb{C}}
\begin{document}
	\newgeometry{left=15mm, top=20mm, right=15mm, bottom=20mm, nohead, nofoot}
	\begin{titlepage}
		\begin{center}
			
			\textbf{МИНИСТЕРСТВО НАУКИ И ВЫСШЕГО ОБРАЗОВАНИЯ}
			
			\textbf{РОССИЙСКОЙ ФЕДЕРАЦИИ}
			
			\textbf{ФЕДЕРАЛЬНОЕ ГОСУДАРСТВЕННОЕ АВТОНОМНОЕ ОБРАЗОВАТЕЛЬНОЕ УЧРЕЖДЕНИЕ ВЫСШЕГО ОБРАЗОВАНИЯ}
			
			\textbf{<<Пермский государственный национальный исследовательский университет>>}
			
			\vspace{5mm}
			
			\textbf{Механико-математический факультет}
			
			\textbf{Кафедра фундаментальной математики}
			
			\vspace{10mm}
			
			\textbf{\large Вихляев Егор Сергеевич} %\\[8mm]
			
			\vspace{10mm}
			% Название
			\textbf{\large ОТЧЁТ О ЛАБОРАТОРНОЙ РАБОТЕ 3 \\ ПО МАШИННОМУ ОБУЧЕНИЮ\\}

			
			\vspace{15mm}
			%Уровень образования: бакалавриат\\
			%Направление 01.03.01 <<Математика>>\\
			%Основная образовательная программа СВ.5005.2015
			%«Прикладная математика, фундаментальная информатика и программирование»\\
			%Профиль «Исследование и проектирование систем управления\\ и обработки сигналов»\\[25mm]
			\begin{flushright}
				\begin{minipage}[t]{0.65\textwidth}
					\raggedleft{{Выполнил:} \\
						студент 3 курса очной формы обучения направления подготовки 01.03.01 -- Механико-математический, направление -- <<Математика>>}
					
					%\vspace{10mm}
					
					%{Рецензент:} \\
					%профессор, кафедра компьютерных технологий \\и систем, д.ф. - м.н. Веремей~Евгений Игоревич
				\end{minipage}
			\end{flushright}
			
			\vspace{5mm}
			% Научный руководитель, рецензент
			
			
			\vfill
			{Пермь}
			\par{\the\year{} г.}
		\end{center}
	\end{titlepage}
	\restoregeometry
	%\setcounter{page}{2}
	\tableofcontents
	\newpage
	
	\chapter*{Введение} % звездочка снимает нумерование главы.
	\addcontentsline{toc}{section}{Введение} % делаем введение видимым в оглавлении
	
	Актуальность: В современном образовании все чаще применяется форма работы, где учащиеся больше занимаются самостоятельной деятельностью, например, научно-исследовательской работой. Это одно из направлений обновления образования, где инициатива зачастую исходит от учителя, который организует и руководит исследованиями учеников, а также несет ответственность за их результаты. Во многих высших учебных заведениях также ожидается, что студенты будут более самостоятельны в своей работе, поэтому научно-исследовательская деятельность становится важным инструментом для развития качеств, таких как самостоятельность, гибкость мышления и любознательность.
	
	Цель: Разработка плана исследования для школьников на тему «История теории гомологий».
	
	Задачи работы:
	\begin{enumerate} 
		\item Изучить проведение научно-исследовательской работы у школьников.
		\item Изучить историю становления теории гомологий.
		\item Разработать план исследования для школьников.
	\end{enumerate}
	
	Объект исследования: школьники 10-11 классов.
	
	Предмет исследования: Теория гомологий.
	
	Место прохождения практики: кафедра фундаментальной математики ПГНИУ.
	
	Сроки прохождения практики: 01.09.2023-25.12.2023
	
	\chapter{Проблема первая: коллинеарные признаки}
	\textbf{Def:} Просмотрев матрицу корреляции, можно заметить, что у нас есть несколько признаков, которые сильно коррелируют друг с другом: temp и atemp (коррелируют по своей природе), два windspeed (разница в ед. измерения). Такие признаки называются коллинеарными. Как увидим в дальнейшем, это негативно сказывается на обучении линейной модели.
	
	Для начала проводим масштабирование или стандартизацию признаков: из каждого признака вычитаем его среднее и делим на стандартное отклонение. Это делается с помощью метода $scale()$.

	Но прежде этого, мы должны перемешать выборку. Это потребуется для дальнейшей кросс-валидации. Это делается методом $shuffle()$.
	
	\textbf{Def: Кросс-валидация} — это метод, предназначенный для оценки качества работы модели, который широко применяется в машинном обучении.	
	
	
	
	\begin{thebibliography}{99}\addcontentsline{toc}{chapter}{Литература}
		\bibitem{fomenko-wife} Борисович Ю.Г., Близняков Н.М., Израилевич Я.А., Фоменко Т.Н. Введение в топологию: Учеб. пособие для вузов. - 1-е изд. - М,: Высш. школа, 1980. - 295 с.
		\bibitem{Vasiliev} Васильев В.А. Топология для младшекурсников. - 2-е изд. - М.: МЦНМО, 2019. - 160 с.
		\bibitem{Vick} Вик Дж. У. Теория гомологий. Введение в алгебраическую топологию. - 1-е изд. - М.: МЦНМО, 2005. - 288 с.
		\bibitem{Dubr, Nov, Fomenko} Дубровин Б.А., Новиков С.П., Фоменко А.Т. Современная геометрия: Методы и приложения. Т.3: Теория гомологий. - 1-е изд. - М.: Эдиториал УРСС, 2001. - 288 с.
		\bibitem{fomenko-fuks} Фоменко А.Т., Фукс Д.Б. Курс гомотопической топологии. - 1-е изд. - М,: Наука. Гл. ред. физ.-мат. лит., 1989. - 528 с.
		\bibitem{Steenrod, Eilenberg} Eilenberg S., Steenrod N. Foundations Of Aglebraic Topology. - 1st изд. - Princeton, New Jersey: Princeton University Press, 1952. - 403 с.
		\bibitem{poincare} Poincare H. Analysis situs. - Ed. 1. - Paris: Journal de l'Ecole Polytechnique, 1895. - 123 с.
	\end{thebibliography}
	
\end{document}